\documentclass{article} \usepackage{amsmath}
\begin{document}
Step 1: Verify $\phi \colon G \to \overline{G}$. (Does the function makes sense? Does it have
the correct domain and range?). It's clear that $\phi(x)=2^{x}$ has a domain of all real numbers.
Does $2^{x}$ stay within the range of $\overline{G}$, the positive real numbers? Yes! This is because
exponents cannot "change" a positive number to a negative number. \[\]
Step 2: Verify $\phi$ is 1-1. Let $x,y \in G$ and suppose $\phi(x)=\phi(y)$. Then
\begin{align*}
2^{x} &= 2^{y} \\
\log_{2}2^{x} &= \log_{2}2^{y} \\
x &= y
\end{align*}
Since $\phi(x) = \phi(y)$ implies $x = y$, $\phi$ is 1-1. \[\]
Step 3: Verify $\phi$ is onto. Let $y \in G$. We must show $\exists x \in G$ such that $\phi(x) = y$. Since
$y \in \overline{G}$, $y$ is a positive, real number. We can find this $x$ by
\begin{align*}
\phi(x) =y \Rightarrow 2^{x} &= y \\
log_{2}(2^{x}) &= log_{2}(y) \\
x &= log_{2}(y)
\end{align*} \[\]
Step 4: Verify $\phi$ preserves the operation. Let $x,y \in G$. Then
\begin{align*}
\phi(xy) &= 2^{x+y} \\
&= 2^{x}2^{y} \\
&= \phi(x)\phi(y)
\end{align*}  
As discussed earlier, when we are attempting to show that $\phi$ is operation-preserving, $\phi(xy)$
applies the operation of $G$ while $\phi(x)\phi(y)$ applies the operation of $\overline{G}$. So, in
this example, the $xy$ in $\phi(xy)$ evaluated to $\phi(x+y)$, which became $2^{x+y}$.


\end{document}
