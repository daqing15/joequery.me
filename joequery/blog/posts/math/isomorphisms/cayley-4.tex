\documentclass{article}
\begin{document}
$\forall g \in G$, define $\phi(g)=T_g$. It's clear that $\phi \colon G \to \overline{G}$. To show $\phi$ is
an isomorphism, we must show that $\phi$ is 1-1, onto, and operation preserving. To verify $\phi$ is 1-1,
suppose $\phi(g) = \phi(h)$, meaning $T_g = T_h$. Then
\begin{align*}
T_g(e) &= T_h(e) \\
ge &= he \\
g &= h
\end{align*}
Since $\phi(g) = \phi(h)$ implies $g=h$, $\phi$ is 1-1. The onto property of $\phi$ is apparent by how
$\phi$ was constructed to map every element $g$ to $T_g$. \[\] 
All that's left to show is that $\phi$ is 
operation-preserving. Let $a,b \in G$. Then
\begin{align}
\phi(ab) &= T_{ab} \\
&= T_aT_b \\
&= \phi(a)\phi(b)
\end{align}
QED.
\end{document}
